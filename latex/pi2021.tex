\documentclass{beamer}
\usepackage{polski}
\usepackage[utf8]{inputenc}
\usepackage{amssymb}
\usepackage{amsmath}
\usepackage{amsthm}
\usepackage{xcolor}
\usepackage{enumitem}
\usefonttheme[onlymath]{serif}
\newcommand*{\theorembreak}{\usebeamertemplate{theorem end}\framebreak\usebeamertemplate{theorem begin}}
\usepackage[ddmmyyyy,hhmmss]{datetime}
\usepackage{animate}
\usepackage{hyperref}
\usepackage{etoolbox}
\patchcmd{\theorem}{Theorem}{Twierdzenie}{}{}
\patchcmd{\corollary}{Corollary}{Wniosek}{}{}
\patchcmd{\lemma}{Lemma}{Lemat}{}{}
\patchcmd{\proposition}{Proposition}{Stwierdzenie}{}{}
\patchcmd{\axiom}{Axiom}{Aksjomat}{}{}
\patchcmd{\example}{Example}{Przykład}{}{}
\patchcmd{\definition}{Definition}{Definicja}{}{}
\patchcmd{\remark}{Remark}{Uwaga}{}{}

\newcommand{\enumsymbol}{$\triangleright$}
\newcommand{\enumsymbolsec}{$\circ$}


\newcommand{\colora}[1]{{\color{red}{#1}}}
\newcommand{\colorb}[1]{{\color{green}{#1}}}
\newcommand{\colorc}[1]{{\color{blue}{#1}}}

\newcommand{\R}{\mathbb{R}}
\newcommand{\Q}{\mathbb{Q}}
\newcommand{\N}{\mathbb{N}}
\newcommand{\Z}{\mathbb{Z}}

\renewcommand{\gcd}[2]{\mathtt{gcd}\paran{#1, #2}}

\newcommand{\convto}[2]{\xrightarrow[#2]{#1}}
\newcommand{\TODO}[1]{(\footnote{\texttt{TODO:} #1}!)}


\let\oldsqrt\sqrt
\def\sqrt{\mathpalette\DHLhksqrt}
\def\DHLhksqrt#1#2{%
\setbox0=\hbox{$#1\oldsqrt{#2\,}$}\dimen0=\ht0
\advance\dimen0-0.2\ht0
\setbox2=\hbox{\vrule height\ht0 depth -\dimen0}%
{\box0\lower0.4pt\box2}}

\newcommand{\define}[1]{\textit{#1}}
\newcommand{\paren}[1]{\!\left(#1 \right)}

\renewcommand*\footnoterule{}

\setbeamertemplate{theorems}[numbered]

\usetheme{Montpellier}

\author{Piotr Idzik}
\title{Co można znaleźć w potęgach dwójki?}
\date{Karlsruhe, Katowice, 15.03.2021}
\institute{blablabla}
\setbeamerfont{page number in head/foot}{size=\large, shape=\ttfamily\color{black}}
%\setbeamertemplate{footline}[frame number]
\beamertemplatenavigationsymbolsempty

\expandafter\def\expandafter\insertshorttitle\expandafter{%
  \insertshorttitle\hfill%
  \texttt{\insertframenumber}}


\begin{document}
\begin{frame}[plain]
\maketitle
%\hfill \texttt{\tiny{ver. \today\ \currenttime\/}}
\hfill \textcolor[rgb]{0.85,0.85,0.85}{\texttt{\tiny{ver. \today\ \currenttime\/}}}
\end{frame}
\section{Sformułowanie problemu}
\subsection{\define{Zawieranie} się liczb}
\begin{frame}
  \begin{definition}
    Niech $a, b \in \N$.
    Mówimy, że liczba $a$ \define{zawiera} liczbę $b$,
    jeżeli reprezentacja dziesiętna liczby $b$ jest podłańcuchem reprezentacji dziesiętnej liczby $a$.
  \end{definition}

  \begin{example}
    \begin{enumerate}[label=\enumsymbol]
      \item liczba 128 zawiera liczby 12 oraz 28,
      \item liczba 128 nie zawiera liczby 18.
    \end{enumerate}
  \end{example}
\end{frame}
\subsection{Potęgi dwójki}
\begin{frame}
  Rozważmy ciąg $(2^n)_{n \in \N}$, tzn. liczby 2, 4, 8, 16, 32, 64, 128, 256, 512, \ldots{}.
  Zauważmy, że
  \begin{enumerate}[label=\enumsymbol]
    \item \colora{28} zawiera się w $2^7 = 1\colora{28}$,
    \item \colora{314} zawiera się w $2^{74} = 188894659\colora{314}78580854784$,
    \item \colora{31415} zawiera się w $2^{144} = 2230074519853062\colora{31415}35718272648361505980416$,
    \item \colora{70000000} zawiera się w $2^{9452}$.
  \end{enumerate}
  Czy prawdą jest, że dla dowolnej liczby $n \in \N$ istnieje taki wykładnik $w \in \N$, że liczba $2^p$ zawiera liczbę $n$?
\end{frame}

\section{Rozstrzygnięcie problemu}
\section{Główne twierdzenie}

\begin{frame}
  \begin{theorem}
    Niech $p \in \N$ będzie takie, że $p \not = 10^r$, $r \in \N$.
    Wówczas dla dowolnej liczby $n \in \N$ istnieje taki wykładnik $w \in \N$, że liczba $p^w$ zawiera liczbę $n$.
  \end{theorem}
  \begin{lemma}
    Niech $p \in \N$ będzie takie, że $\log_{10}p \in \R \setminus \Q$.
    Wówczas dla dowolnej liczby $n \in \N$ istnieje taki wykładnik $w \in \N$, że liczba $p^w$ rozpoczyna się liczbą $n$. 
  \end{lemma}
  \begin{lemma}
    $\log_{10}p \in \Q$ wtedy i tylko wtedy, gdy $p = 10^r$, dla pewnego $r \in \N$.
  \end{lemma}
\end{frame}
\end{document}
