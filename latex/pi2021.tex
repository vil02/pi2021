\documentclass[aspectratio=169]{beamer}
\usepackage{polski}
\usepackage[utf8]{inputenc}
\usepackage{amssymb}
\usepackage{amsmath}
\usepackage{amsthm}
\usepackage{xcolor}
\usepackage{enumitem}
\usefonttheme[onlymath]{serif}
\newcommand*{\theorembreak}{\usebeamertemplate{theorem end}\framebreak\usebeamertemplate{theorem begin}}
\usepackage[ddmmyyyy,hhmmss]{datetime}
\usepackage{animate}
\usepackage{seqsplit}
\usepackage{fontawesome}
\usepackage{import}
\usepackage{animate}
\usepackage[noadjust]{cite}
\usepackage{hyperref}
\usepackage{etoolbox}
\patchcmd{\theorem}{Theorem}{Twierdzenie}{}{}
%\patchcmd{\corollary}{Corollary}{Wniosek}{}{}
\patchcmd{\lemma}{Lemma}{Lemat}{}{}
%\patchcmd{\proposition}{Proposition}{Stwierdzenie}{}{}
%\patchcmd{\axiom}{Axiom}{Aksjomat}{}{}
\patchcmd{\example}{Example}{Przykład}{}{}
\patchcmd{\definition}{Definition}{Definicja}{}{}
%\patchcmd{\remark}{Remark}{Uwaga}{}{}

\newcommand{\enumsymbol}{$\triangleright$}
\newcommand{\enumsymbolsec}{$\circ$}


\newcommand{\colora}[1]{{\color{red}{#1}}}

\newcommand{\R}{\mathbb{R}}
\newcommand{\Q}{\mathbb{Q}}
\newcommand{\N}{\mathbb{N}}
\newcommand{\Z}{\mathbb{Z}}

\renewcommand{\gcd}[2]{\mathtt{NWD}\paren{#1, #2}}

\newcommand{\convto}[2]{\xrightarrow[#2]{#1}}
\newcommand{\TODO}[1]{(\footnote{\texttt{TODO:} #1}!)}

\newcommand{\myEmail}{vil02@o2.pl}
\newcommand{\myLinkedinLink}{https://www.linkedin.com/in/piotr-idzik-34b572151/}
\newcommand{\myGithubLink}{https://github.com/vil02/}

\newcommand{\myLinkedin}{\href{\myLinkedinLink}{\faLinkedinSquare}}
\newcommand{\myGithub}{\href{\myGithubLink}{\faGithubSquare}}

\let\oldsqrt\sqrt{}
\def\sqrt{\mathpalette\DHLhksqrt}
\def\DHLhksqrt#1#2{%
\setbox0=\hbox{$#1\oldsqrt{#2\,}$}\dimen0=\ht0
\advance\dimen0-0.2\ht0 % chktex 8
\setbox2=\hbox{\vrule height\ht0 depth -\dimen0}%
{\box0\lower0.4pt\box2}}

\newcommand{\labelOnSlide}[2]{\label<#1>{#2}}
\definecolor{colorOfDefinition}{RGB}{0, 70, 230}
\newcommand{\define}[1]{\textcolor{colorOfDefinition}{\textit{#1}}}
\newcommand{\paren}[1]{\!\left(#1 \right)}
\newcommand{\floor}[1]{\left\lfloor{} #1 \right\rfloor}
\newcommand{\fracpart}[1]{\left\{ #1 \right\}}
\newcommand{\abs}[1]{\left| #1 \right|}
\newcommand{\email}[1]{\href{mailto: #1}{\texttt{#1}}}
\newcommand{\link}[1]{\href{#1}{\texttt{#1}}}
\newcommand{\goToProof}[1]{\hfill\hyperlink{#1}{\normalfont\faArrowCircleRight}}
\renewcommand*\footnoterule{}

\setbeamertemplate{theorems}[numbered]

\newcommand{\tmpDataFolder}{../tmp_data/} % folder path for all python generated data; relative to latex main file
\newcommand{\expSubstrPlotsTex}{exp_substr_plots.tex}
\newcommand{\wheelsRationalTex}{wheels_rational.tex}
\newcommand{\wheelsIrrationalTex}{wheels_irrational.tex}
\newcommand{\fracPartsOfRationalMultiplesTex}{fracparts_of_rational_multiples.tex}
\newcommand{\fracPartsOfIrationalMultiplesTex}{fracparts_of_irrational_multiples.tex}


\usetheme{Montpellier}
\author{\texorpdfstring{\href{\myLinkedinLink}{dr Piotr Idzik} \\ \email{\myEmail}}{dr Piotr Idzik}}
\title{Co można znaleźć w potęgach dwójki?}
\date{Karlsruhe, Katowice, 15.03.2021}
\setbeamerfont{page number in head/foot}{size=\large, shape=\ttfamily\color{black}}

\setbeamertemplate{navigation symbols}
{%
  \vbox{%
  \hbox{\insertbackfindforwardnavigationsymbol}}%
}

\expandafter\def\expandafter\insertshorttitle\expandafter{%
  \insertshorttitle\hfill%
  \texttt{\insertframenumber}}

\begin{document}
\begin{frame}[plain]
\maketitle

\myLinkedin{}
\myGithub{}
\hfill \textcolor[rgb]{0.85,0.85,0.85}{\texttt{\tiny{ver. \today\ \currenttime\/}}}
\end{frame}
\section{Sformułowanie problemu}
\subsection{\define{Zawieranie} się liczb}
\begin{frame}
  \begin{definition}
    Niech $a, b \in \N$.
    Mówimy, że liczba $a$ \define{zawiera} liczbę $b$,
    jeżeli reprezentacja dziesiętna liczby $b$ jest podłańcuchem reprezentacji dziesiętnej liczby $a$.
  \end{definition}
  \pause{}
  \begin{example}
    \begin{enumerate}[label=\enumsymbol]
      \item<+-> liczba 128 zawiera liczby 12 oraz 28,
      \item<+-> liczba 128 nie zawiera liczby 18.
    \end{enumerate}
  \end{example}
\end{frame}
\subsection{Potęgi dwójki}
\begin{frame}
  Rozważmy ciąg ${(2^n)}_{n \in \N}$, tzn.\ liczby 2, 4, 8, 16, 32, 64, 128, 256, 512, 1024, 2048, \ldots{}.
  \pause{}
  \begin{enumerate}[label=\enumsymbol]
    \item<+-> \colora{28} zawiera się w $2^7 = 1\colora{28}$,
    \item<+-> \colora{314} zawiera się w $2^{74} = 188894659\colora{314}78580854784$,
    \item<+-> \colora{31415} zawiera się w $2^{144} = 2230074519853062\colora{31415}35718272648361505980416$,
    \item<+-> 70000000 zawiera się w $2^{9452}$,
    \item<+-> 2000000000 zawiera się w $2^{100824}$.
  \end{enumerate}
  \onslide<+-> Czy prawdą jest, że dla dowolnej liczby $n \in \N$ istnieje taki wykładnik $w \in \N$, że liczba $2^w$ zawiera liczbę $n$?
\end{frame}

\section{Symulacje komputerowe}

\subimport{\tmpDataFolder}{\expSubstrPlotsTex}

\section{Rozstrzygnięcie problemu}
\section{Główne twierdzenie}

\begin{frame}
  \pause{}
  \begin{theorem}
    Niech $p \in \N$ będzie takie, że $p \not = 10^r$, $r \in \N_0$.
    Wówczas dla dowolnej liczby $n \in \N$ istnieje taki wykładnik $w \in \N$, że liczba $p^w$ zawiera liczbę $n$.
  \end{theorem}
  \pause{}
  \begin{lemma}
    \labelOnSlide{4}{lemma::powers_begin_with_every_string}
    Niech $p \in \N$ będzie takie, że $\log_{10}p \in \R \setminus \Q$.
    Wówczas dla dowolnej liczby $n \in \N$ istnieje taki wykładnik $w \in \N$, że liczba $p^w$ \define{zaczyna się} liczbą $n$. \goToProof{proof::lemma::powers_begin_with_every_string}
  \end{lemma}
  \pause{}
  \begin{lemma}
    \labelOnSlide{4}{lemma::log_10_rational_iff}
    $\log_{10}p \in \Q$ wtedy i tylko wtedy, gdy $p = 10^r$, dla pewnego $r \in \N_0$.
    \goToProof{proof::lemma::log_10_rational_iff}
  \end{lemma}
\end{frame}

\section{\texorpdfstring{Przygotowania do dowodu lematu~\protect\ref{lemma::powers_begin_with_every_string}}{Przygotowania do dowodu głównego lematu}}
\subsection{Liczby wymierne i niewymierne}

\begin{frame}
  \begin{figure}
    \subimport{\tmpDataFolder}{\wheelsRationalTex}
  \end{figure}
  \begin{theorem}
  Jeżeli $\frac{r_A}{r_B} = \frac{a}{b}$, $\gcd{a}{b} = 1$, to po wykonaniu $b$ obrotów przez koło $A$ ($a$ obrotów przez koło $B$)
  \textcolor{\wheelMarkerColor}{znaczniki} będą w tym samym położeniu.
  \end{theorem}
\end{frame}

\begin{frame}
  \begin{figure}
    \subimport{\tmpDataFolder}{\wheelsIrrationalTex}
  \end{figure}
\end{frame}

\subsection{Część ułamkowa liczby}

\begin{frame}
\begin{definition}
Niech $x \in \R$.
\define{Częścią całkowitą liczby} $x$ nazywamy największą liczbę całkowitą nie większą niż $x$ i oznaczamy symbolem $\floor{x}$.

\uncover<3->{\define{Część ułamkową liczby} $x$ definiujemy wzorem
\begin{equation}
\labelOnSlide{6-}{eq::frac_part_definition}
\fracpart{x} = x-\floor{x} \uncover<4->{ \in \left[0, 1\right).}
\end{equation}}
\end{definition}
\uncover<2->{
\begin{example}
  \begin{enumerate}[label=\enumsymbol]
    \item<2-> $\floor{\pi} = 3$, \uncover<5->{$\fracpart{\pi} = 0{,}14159265\ldots$,}
    \item<2-> $\floor{15} = 15$, \uncover<5->{$\fracpart{15} = 0$,}
    \item<6-> $\floor{-4{,}25} = -5$, $\fracpart{-4{,}25} = 0{,}75$.
  \end{enumerate}
\end{example}}
\end{frame}

\subsection{\texorpdfstring{O ciągach $\paren{\fracpart{n\gamma}}_{n \in \N}$}{O ciągach wielokrotności liczb mod 1}}

\subimport{\tmpDataFolder}{\fracPartsOfRationalMultiplesTex}

\begin{frame}
  \begin{example}
    Rozważmy ciąg ${\paren{\fracpart{n\pi}}}_{n \in \N}$.
    \begin{figure}
      \subimport{\tmpDataFolder}{\fracPartsOfIrrationalMultiplesTex}
    \end{figure}
  \end{example}

  \begin{theorem}[\cite{Sierpinski1910, Weyl1916}]
    \label{thm::fractional_part_of_multiples_of_irrational_numers_are_dense}
    Jeżeli $\gamma \in \R \setminus \Q$, to ciąg $\paren{\fracpart{n\gamma}}_{n \in \N}$ jest \define{gęsty} w przedziale $[0, 1]$,
    tzn.\ dowolny punkt przedziału $[0, 1]$ może zostać dowolnie dokładnie przybliżony wyrazami ciągu $\paren{\fracpart{n\gamma}}_{n \in \N}$.
  \end{theorem}
\end{frame}

\subsection{Logarytmy dziesiętne}

\begin{frame}
  \begin{equation*}
    10^w = s \iff \log_{10}s = w,
  \end{equation*}
  \pause{}
  \begin{enumerate}[label=\enumsymbol]
    \item<+-> $\log_{10}100 = 2$, bo $10^2 = 100$,
    \item<+-> $\log_{10}10^\alpha = \alpha \quad \paren{\alpha \in \R}$,
    \item<+-> $\log_{10}2 = 0{,}30102\ldots, \log_{10}3 = 0{,}47712\ldots, \log_{10}4 = 0{,}60205\ldots \uncover<+->{\stackrel{\text{Lemat~\ref{lemma::log_10_rational_iff}}}{\in} \R \setminus \Q$,
    \item<+-> $10^{\log_{10}\alpha} = \alpha \quad \paren{\alpha \in \R}$,}
  \end{enumerate}
  \uncover<+->{\begin{equation} \labelOnSlide{10-}{eq::log_is_inverse_to_exp}
    10^{\log_{10}p^w} = p^w \quad \paren{p, w \in \N},
  \end{equation}}
  \uncover<+->{\begin{equation} \labelOnSlide{10-}{eq::exp_of_sum}
     10^{w_1+w_2} = 10^{w_1} \cdot 10^{w_2} \quad \paren{w_1, w_2 \in \R},
  \end{equation}}
  \begin{enumerate}[label=\enumsymbol]
    \item<+-> $\log_{10} x_1 x_2 = \log_{10}x_1+\log_{10}x_2 \quad \paren{x_1, x_2 > 0}$,
  \end{enumerate}
  \uncover<+->{\begin{equation} \labelOnSlide{10-}{eq::log_of_pow}
    \log_{10}{p^w} = w \log_{10}p  \quad \paren{p, w \in \N}.
  \end{equation}}
\end{frame}

\section{\texorpdfstring{Dowód lematu~\protect\ref{lemma::powers_begin_with_every_string}}{Dowód głównego lematu}}

\begin{frame}
  \begin{proof}[Dowód lematu~\ref{lemma::powers_begin_with_every_string}]
    \hypertarget{proof::lemma::powers_begin_with_every_string}{}
    \pause{}
    \begin{align*}
      \uncover<+->{p^w} & \uncover<+->{\stackrel{\eqref{eq::log_is_inverse_to_exp}}{=} 10^{\log_{10}p^w}}
            \uncover<+->{\stackrel{\eqref{eq::log_of_pow}}{=} 10^{w \log_{10} p}}
            \uncover<+->{\stackrel{\eqref{eq::frac_part_definition}}{=} 10^{\floor{w \log_{10} p} + \fracpart{w \log_{10} p}}} \\
          & \uncover<+->{\stackrel{\eqref{eq::exp_of_sum}}{=} \underbrace{10^{\floor{w \log_{10} p}}}_{10\ldots 0}\cdot10^{\fracpart{w \log_{10} p}}.}
    \end{align*}
    \begin{enumerate}[label=\enumsymbol]
      \item<+-> $10^{\log_{10} 1{,}2345} = 1{,}2345$,
      \item<+-> $10^{\log_{10} c_1{,}c_2c_3c_3 \ldots c_l} = c_1{,}c_2c_3c_3 \ldots c_l$.
    \end{enumerate}
    \onslide<+-> Niech $n = c_1c_2c_3 \ldots c_l$,
    \onslide<+-> wobec twierdzenia~\ref{thm::fractional_part_of_multiples_of_irrational_numers_are_dense} istnieje $w > \frac{l}{\log_{10}p}$ takie, że
    $\fracpart{w \log_{10} p}$ jest \textit{wystarczająco dobrym} przybliżeniem $\log_{10}{c_1{,}c_2c_3c_3 \ldots c_l}$.
  \end{proof}

\end{frame}

\section{Uwagi końcowe}
\begin{frame}
  \begin{theorem}
    Jeżeli $\paren{{\fracpart{\log_{10} a_k}}}_{k \in \N}$ jest gęsty w przedziale  $[0, 1]$, to
    ciąg ${\paren{a_k}}_{k \in \N}$ \define{zawiera wszystkie liczby}.
  \end{theorem}
  \pause{}
  \begin{example}
  \begin{enumerate}[label=\enumsymbol]
    \item<+-> $\paren{\fracpart{n\log_{10}2}}_{n \in \N} \rightsquigarrow \paren{2^n}_{n \in \N}$,
    \item<+-> $\paren{\fracpart{n^k\log_{10}2}}_{n \in \N} \rightsquigarrow \paren{2^{n^k}}_{n \in \N}$, dla dowolnego $k \in \N$,
    \item<+-> $\paren{\fracpart{p_n\log_{10}2}}_{n \in \N} \rightsquigarrow \paren{2^{p_n}}_{n \in \N}$, gdzie $p_n$ oznacza $n$-tą liczbę pierwszą (zob.~\cite{Vinogradov1948}),
    \item<+-> $\paren{\fracpart{k\log_{10}n}}_{n \in \N} \rightsquigarrow \paren{n^k}_{n \in \N}$, dla dowolnego $k \in \N$,
    \item<+-> $\paren{\fracpart{\log_{10}n!}}_{n \in \N} \rightsquigarrow\paren{n!}_{n \in \N}$ (zob.~\cite{Diaconis1977}),
  \end{enumerate}
  \end{example}
  \onslide<+-> Prezentacja: \link{\myGithubLink pi2021/}
\end{frame}

\appendix

\section{Symulacje komputerowe dla innych ciągów}

\subimport{\tmpDataFolder}{\powSubstrPlotsTex}

\subsection{\texorpdfstring{$n! = 1\cdot2\cdot \ldots \cdot n, \quad \paren{n \in \N}$, $0! = 1$}{Silnia}} % chktex 11

\subimport{\tmpDataFolder}{\factorialSubstrPlotsTex}


\section{\texorpdfstring{Dowód lematu~\protect\ref{lemma::log_10_rational_iff}}{Dowód warunku równoważnego wymierności logarytmu dziesiętnego}}

\begin{frame}
  \begin{proof}[Dowód lematu~\ref{lemma::log_10_rational_iff}]
    \hypertarget{proof::lemma::log_10_rational_iff}{}
    Załóżmy, że $\log_{10}p = \frac{a}{b}$, dla pewnych $a, b \in \N$.
    Wówczas $10^{\frac{a}{b}} = p$, stąd $10^a = p^b$, i ponadto $2^a5^a = p^b$.
    Zatem $p = 2^r5^s$, dla pewnych $r, s \in \N$.
    Pozostaje pokazać, że $r = s$.
    Istotnie, $2^{rb}5^{sb} = 2^a5^a$, skąd $a = rb = sb$, więc $r=s$.
    A zatem $p = 2^r5^r = 10^r$.
  \end{proof}
\end{frame}

\section{Bibliografia}
\begin{frame}
  \bibliography{bib_data}{}
  \bibliographystyle{plain}
\end{frame}

\end{document}
