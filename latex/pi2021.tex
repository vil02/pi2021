\documentclass{beamer}
\usepackage{polski}
\usepackage[utf8]{inputenc}
\usepackage{amssymb}
\usepackage{amsmath}
\usepackage{amsthm}
\usepackage{xcolor}
\usepackage{enumitem}
\usefonttheme[onlymath]{serif}
\newcommand*{\theorembreak}{\usebeamertemplate{theorem end}\framebreak\usebeamertemplate{theorem begin}}
\usepackage[ddmmyyyy,hhmmss]{datetime}
\usepackage{animate}
\usepackage{hyperref}
\usepackage{etoolbox}
\patchcmd{\theorem}{Theorem}{Twierdzenie}{}{}
\patchcmd{\corollary}{Corollary}{Wniosek}{}{}
\patchcmd{\lemma}{Lemma}{Lemat}{}{}
\patchcmd{\proposition}{Proposition}{Stwierdzenie}{}{}
\patchcmd{\axiom}{Axiom}{Aksjomat}{}{}
\patchcmd{\example}{Example}{Przykład}{}{}
\patchcmd{\definition}{Definition}{Definicja}{}{}
\patchcmd{\remark}{Remark}{Uwaga}{}{}

\newcommand{\enumsymbol}{$\triangleright$}
\newcommand{\enumsymbolsec}{$\circ$}


\newcommand{\colora}[1]{{\color{red}{#1}}}
\newcommand{\colorb}[1]{{\color{green}{#1}}}
\newcommand{\colorc}[1]{{\color{blue}{#1}}}

\newcommand{\R}{\mathbb{R}}
\newcommand{\Q}{\mathbb{Q}}
\newcommand{\N}{\mathbb{N}}
\newcommand{\Z}{\mathbb{Z}}

\renewcommand{\gcd}[2]{\mathtt{gcd}\paran{#1, #2}}

\newcommand{\convto}[2]{\xrightarrow[#2]{#1}}
\newcommand{\TODO}[1]{(\footnote{\texttt{TODO:} #1}!)}


\let\oldsqrt\sqrt
\def\sqrt{\mathpalette\DHLhksqrt}
\def\DHLhksqrt#1#2{%
\setbox0=\hbox{$#1\oldsqrt{#2\,}$}\dimen0=\ht0
\advance\dimen0-0.2\ht0
\setbox2=\hbox{\vrule height\ht0 depth -\dimen0}%
{\box0\lower0.4pt\box2}}

\newcommand{\define}[1]{\textit{#1}}
\newcommand{\paren}[1]{\!\left(#1 \right)}
\newcommand{\floor}[1]{\left\lfloor #1 \right\rfloor}
\newcommand{\fracpart}[1]{\left\{ #1 \right\}}
\newcommand{\abs}[1]{\left| #1 \right|}


\renewcommand*\footnoterule{}

\setbeamertemplate{theorems}[numbered]

\usetheme{Montpellier}

\author{Piotr Idzik}
\title{Co można znaleźć w potęgach dwójki?}
\date{Karlsruhe, Katowice, 15.03.2021}
\institute{blablabla}
\setbeamerfont{page number in head/foot}{size=\large, shape=\ttfamily\color{black}}
%\setbeamertemplate{footline}[frame number]
\beamertemplatenavigationsymbolsempty

\expandafter\def\expandafter\insertshorttitle\expandafter{%
  \insertshorttitle\hfill%
  \texttt{\insertframenumber}}


\begin{document}
\begin{frame}[plain]
\maketitle
%\hfill \texttt{\tiny{ver. \today\ \currenttime\/}}
\hfill \textcolor[rgb]{0.85,0.85,0.85}{\texttt{\tiny{ver. \today\ \currenttime\/}}}
\end{frame}
\section{Sformułowanie problemu}
\subsection{\define{Zawieranie} się liczb}
\begin{frame}
  \begin{definition}
    Niech $a, b \in \N$.
    Mówimy, że liczba $a$ \define{zawiera} liczbę $b$,
    jeżeli reprezentacja dziesiętna liczby $b$ jest podłańcuchem reprezentacji dziesiętnej liczby $a$.
  \end{definition}

  \begin{example}
    \begin{enumerate}[label=\enumsymbol]
      \item liczba 128 zawiera liczby 12 oraz 28,
      \item liczba 128 nie zawiera liczby 18.
    \end{enumerate}
  \end{example}
\end{frame}
\subsection{Potęgi dwójki}
\begin{frame}
  Rozważmy ciąg $(2^n)_{n \in \N}$, tzn. liczby 2, 4, 8, 16, 32, 64, 128, 256, 512, \ldots{}.
  Zauważmy, że
  \begin{enumerate}[label=\enumsymbol]
    \item \colora{28} zawiera się w $2^7 = 1\colora{28}$,
    \item \colora{314} zawiera się w $2^{74} = 188894659\colora{314}78580854784$,
    \item \colora{31415} zawiera się w $2^{144} = 2230074519853062\colora{31415}35718272648361505980416$,
    \item \colora{70000000} zawiera się w $2^{9452}$.
  \end{enumerate}
  Czy prawdą jest, że dla dowolnej liczby $n \in \N$ istnieje taki wykładnik $w \in \N$, że liczba $2^p$ zawiera liczbę $n$?
\end{frame}

\section{Rozstrzygnięcie problemu}
\section{Główne twierdzenie}

\begin{frame}
  \begin{theorem}
    Niech $p \in \N$ będzie takie, że $p \not = 10^r$, $r \in \N$.
    Wówczas dla dowolnej liczby $n \in \N$ istnieje taki wykładnik $w \in \N$, że liczba $p^w$ zawiera liczbę $n$.
  \end{theorem}
  \begin{lemma}
    \label{lemma::powers_beginb_with_every_string}
    Niech $p \in \N$ będzie takie, że $\log_{10}p \in \R \setminus \Q$.
    Wówczas dla dowolnej liczby $n \in \N$ istnieje taki wykładnik $w \in \N$, że liczba $p^w$ rozpoczyna się liczbą $n$. 
  \end{lemma}
  \begin{lemma}
    \label{lemma::log_10_irrational_iff}
    $\log_{10}p \in \Q$ wtedy i tylko wtedy, gdy $p = 10^r$, dla pewnego $r \in \N$.
  \end{lemma}
\end{frame}

\begin{frame}
\begin{definition}
Niech $x \in \R$.
\define{Częścią całkowitą liczby} $x$ nazywamy największą liczbę całkowitą nie większą niż $x$ i oznaczamy symbolem $\floor{x}$.

\define{Część ułamkową liczby} $x$ definiujemy wzorem
\begin{equation*}
\fracpart{x} = x-\floor{x} \in \left[0, 1\right).
\end{equation*}
\end{definition}
\begin{example}
  \begin{enumerate}[label=\enumsymbol]
    \item $\floor{\pi} = 3$, $\fracpart{\pi} = 0.14159265\ldots{}$,
    \item $\floor{15} = 15$, $\fracpart{15} = 0$,
    \item $\floor{-4.25} = -5$, $\fracpart{-4.25} = 0.75$, 
  \end{enumerate}
\end{example}
\end{frame}

\begin{frame}
  $$10^w = x \iff \log_{10}x = w,$$
  \begin{enumerate}[label=\enumsymbol]
    \item $\log_{10}100 = 2$, bo $10^2 = 100$,
    \item $\log_{10}10^\alpha = \alpha \quad \paren{\alpha \in \R}$,
    \item $\log_{10}2 = 0,30102\ldots{} \stackrel{\text{Lemat~\ref{lemma::log_10_irrational_iff}}}{\in} \R \setminus \Q$,
    \item $\log_{10}3 = 0,47712\ldots{} \stackrel{\text{Lemat~\ref{lemma::log_10_irrational_iff}}}{\in} \R \setminus \Q$,
    \item $\log_{10}4 = 0,60205\ldots{} \stackrel{\text{Lemat~\ref{lemma::log_10_irrational_iff}}}{\in} \R \setminus \Q$,
    \item $10^{\log_{10}\alpha} = \alpha \quad \paren{\alpha \in \R}$,
    \item $10^{\log_{10}p^w} = p^w \quad \paren{p, w \in \N}$,
    \item $10^{w_1+w_2} = 10^{w_1} \cdot 10^{w_2} \quad \paren{w_1, w_2 \in \R}$,
    \item $\log_{10} x_1 x_2 = \log_{10}x_1+\log_{10}x_2 \quad \paren{x_1, x_2 > 0}$,
    \item $\log_{10}{p^w} = w \log_{10}p  \quad \paren{w_1, w_2 \in \R}$
  \end{enumerate}
\end{frame}

\begin{frame}
  \begin{theorem}
    \label{thm::fractional_part_of_multiples_of_irrational_numers_are_dense}
    Jeżeli $\gamma \in \R \setminus \Q$, to ciąg $\paren{\fracpart{n\gamma}}_{n \in \N}$ jest \define{gęsty} w przedziale $[0, 1]$,
    tzn. dowolny punkt przedziału $[0, 1]$ może zostać dowolnie dokładnie przybliżony wyrazami ciągu $\paren{\fracpart{n\gamma}}_{n \in \N}$.
  \end{theorem}
\end{frame}


\begin{frame}
  \begin{proof}[Dowód lematu~\ref{lemma::powers_beginb_with_every_string}]
    \begin{align*}
      p^w &= 10^{\log_{10}p^w} = 10^{w \log_{10} p} = 10^{\floor{w \log_{10} p} + \fracpart{w \log_{10} p}} \\
          &= \underbrace{10^{\floor{w \log_{10} p}}}_{10\ldots 0}\cdot10^{\fracpart{w \log_{10} p}}.
    \end{align*}
    \begin{enumerate}[label=\enumsymbol]
      \item $10^{\log_{10} 1.2345} = 1.2345$,
      \item $10^{\log_{10} c_1,c_2c_3c_3 \ldots{} c_l} = c_1,c_2c_3c_3 \ldots{} c_l$.
    \end{enumerate}
    Niech $n = c_1c_2c_3 \ldots{} c_l$, wobec twierdzenia~\ref{thm::fractional_part_of_multiples_of_irrational_numers_are_dense} istnieje $w > \frac{l}{\log_{10}p}$ takie, że
    $\fracpart{w \log_{10} p}$ jest \textit{wystarczająco dobrym} przybliżeniem $\log_{10}{c_1,c_2c_3c_3 \ldots{} c_l}$.  
  \end{proof}
  
\end{frame}

\end{document}
